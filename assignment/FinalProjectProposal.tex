% Options for packages loaded elsewhere
% Options for packages loaded elsewhere
\PassOptionsToPackage{unicode}{hyperref}
\PassOptionsToPackage{hyphens}{url}
\PassOptionsToPackage{dvipsnames,svgnames,x11names}{xcolor}
%
\documentclass[
  letterpaper,
  DIV=11,
  numbers=noendperiod]{scrartcl}
\usepackage{xcolor}
\usepackage{amsmath,amssymb}
\setcounter{secnumdepth}{-\maxdimen} % remove section numbering
\usepackage{iftex}
\ifPDFTeX
  \usepackage[T1]{fontenc}
  \usepackage[utf8]{inputenc}
  \usepackage{textcomp} % provide euro and other symbols
\else % if luatex or xetex
  \usepackage{unicode-math} % this also loads fontspec
  \defaultfontfeatures{Scale=MatchLowercase}
  \defaultfontfeatures[\rmfamily]{Ligatures=TeX,Scale=1}
\fi
\usepackage{lmodern}
\ifPDFTeX\else
  % xetex/luatex font selection
\fi
% Use upquote if available, for straight quotes in verbatim environments
\IfFileExists{upquote.sty}{\usepackage{upquote}}{}
\IfFileExists{microtype.sty}{% use microtype if available
  \usepackage[]{microtype}
  \UseMicrotypeSet[protrusion]{basicmath} % disable protrusion for tt fonts
}{}
\makeatletter
\@ifundefined{KOMAClassName}{% if non-KOMA class
  \IfFileExists{parskip.sty}{%
    \usepackage{parskip}
  }{% else
    \setlength{\parindent}{0pt}
    \setlength{\parskip}{6pt plus 2pt minus 1pt}}
}{% if KOMA class
  \KOMAoptions{parskip=half}}
\makeatother
% Make \paragraph and \subparagraph free-standing
\makeatletter
\ifx\paragraph\undefined\else
  \let\oldparagraph\paragraph
  \renewcommand{\paragraph}{
    \@ifstar
      \xxxParagraphStar
      \xxxParagraphNoStar
  }
  \newcommand{\xxxParagraphStar}[1]{\oldparagraph*{#1}\mbox{}}
  \newcommand{\xxxParagraphNoStar}[1]{\oldparagraph{#1}\mbox{}}
\fi
\ifx\subparagraph\undefined\else
  \let\oldsubparagraph\subparagraph
  \renewcommand{\subparagraph}{
    \@ifstar
      \xxxSubParagraphStar
      \xxxSubParagraphNoStar
  }
  \newcommand{\xxxSubParagraphStar}[1]{\oldsubparagraph*{#1}\mbox{}}
  \newcommand{\xxxSubParagraphNoStar}[1]{\oldsubparagraph{#1}\mbox{}}
\fi
\makeatother


\usepackage{longtable,booktabs,array}
\usepackage{calc} % for calculating minipage widths
% Correct order of tables after \paragraph or \subparagraph
\usepackage{etoolbox}
\makeatletter
\patchcmd\longtable{\par}{\if@noskipsec\mbox{}\fi\par}{}{}
\makeatother
% Allow footnotes in longtable head/foot
\IfFileExists{footnotehyper.sty}{\usepackage{footnotehyper}}{\usepackage{footnote}}
\makesavenoteenv{longtable}
\usepackage{graphicx}
\makeatletter
\newsavebox\pandoc@box
\newcommand*\pandocbounded[1]{% scales image to fit in text height/width
  \sbox\pandoc@box{#1}%
  \Gscale@div\@tempa{\textheight}{\dimexpr\ht\pandoc@box+\dp\pandoc@box\relax}%
  \Gscale@div\@tempb{\linewidth}{\wd\pandoc@box}%
  \ifdim\@tempb\p@<\@tempa\p@\let\@tempa\@tempb\fi% select the smaller of both
  \ifdim\@tempa\p@<\p@\scalebox{\@tempa}{\usebox\pandoc@box}%
  \else\usebox{\pandoc@box}%
  \fi%
}
% Set default figure placement to htbp
\def\fps@figure{htbp}
\makeatother





\setlength{\emergencystretch}{3em} % prevent overfull lines

\providecommand{\tightlist}{%
  \setlength{\itemsep}{0pt}\setlength{\parskip}{0pt}}



 


\KOMAoption{captions}{tableheading}
\makeatletter
\@ifpackageloaded{caption}{}{\usepackage{caption}}
\AtBeginDocument{%
\ifdefined\contentsname
  \renewcommand*\contentsname{Table of contents}
\else
  \newcommand\contentsname{Table of contents}
\fi
\ifdefined\listfigurename
  \renewcommand*\listfigurename{List of Figures}
\else
  \newcommand\listfigurename{List of Figures}
\fi
\ifdefined\listtablename
  \renewcommand*\listtablename{List of Tables}
\else
  \newcommand\listtablename{List of Tables}
\fi
\ifdefined\figurename
  \renewcommand*\figurename{Figure}
\else
  \newcommand\figurename{Figure}
\fi
\ifdefined\tablename
  \renewcommand*\tablename{Table}
\else
  \newcommand\tablename{Table}
\fi
}
\@ifpackageloaded{float}{}{\usepackage{float}}
\floatstyle{ruled}
\@ifundefined{c@chapter}{\newfloat{codelisting}{h}{lop}}{\newfloat{codelisting}{h}{lop}[chapter]}
\floatname{codelisting}{Listing}
\newcommand*\listoflistings{\listof{codelisting}{List of Listings}}
\makeatother
\makeatletter
\makeatother
\makeatletter
\@ifpackageloaded{caption}{}{\usepackage{caption}}
\@ifpackageloaded{subcaption}{}{\usepackage{subcaption}}
\makeatother
\usepackage{bookmark}
\IfFileExists{xurl.sty}{\usepackage{xurl}}{} % add URL line breaks if available
\urlstyle{same}
\hypersetup{
  pdftitle={Final Project Proposal},
  pdfauthor={Ellen Bledsoe},
  colorlinks=true,
  linkcolor={blue},
  filecolor={Maroon},
  citecolor={Blue},
  urlcolor={Blue},
  pdfcreator={LaTeX via pandoc}}


\title{Final Project Proposal}
\author{Ellen Bledsoe}
\date{}
\begin{document}
\maketitle


\section{Final Project Proposal}\label{final-project-proposal}

Answering the questions in this document are worth 20 points for
Question 5 in Week 9's Assignment.

\subsection{Where to Find Data}\label{where-to-find-data}

\textbf{NOTE: If you have your own data from a research project, use
it!}

If you don't have your own data, here are some suggestions for where to
find data:

\begin{enumerate}
\def\labelenumi{\arabic{enumi}.}
\item
  From your lab group

  If you are working in a lab group, you can almost certainly find data
  to work with from your lab! Ask your PI or other lab members if you
  can use their data for this project. It is often helpful to have data
  that relates somewhat to a project or topic you are familiar with.
\item
  From the ``Data is Plural'' Archive

  \href{https://www.data-is-plural.com/}{Data is Plural} is a weekly
  newsletter about complied datasets that people have found. They have
  an archive of all of the datasets that have been included in the
  newsletters, which is nearly 2000.

  Just because the datasets have been compiled does not mean that they
  are clean! I've used a few of them in class as demonstrations.
\item
  From an agency, NGO, or non-profit

  If you have contacts in a local, state, or federal agency or
  non-profit, they likely have some languishing data floating around
  that you can clean up.
\item
  A data repository

  There are many data repositories where you can find data. I would
  recommend sticking to some of the other options above to keep your
  search a bit more constrained, but feel free to ultimately use any
  dataset you find that you would like to use.
\item
  From me!

  If you aren't having any success finding a dataset, let me know. I
  have a few I can share with you, if needed.
\end{enumerate}

\subsection{Project Guidelines}\label{project-guidelines}

As a reminder, the general guidelines for the final project that the
project will need to include core concepts covered in a certain number
of weeks (6 for RNR 437, 10 for RNR 537), bringing together many aspects
of what we will have learned through the course of the semester.

The final project for this course is worth 300 points (RNR 437) or 500
points (RNR 537).

\subsection{Questions}\label{questions}

The following answers are not binding! You can change your final project
as much as you'd like from what you propose here.

Answering these questions is meant to get you thinking about what data
you would like to use and what tasks you might perform with the data.

Note: Your data set does not need to be ecological in nature if you have
another interest. It should also not be a dataset that we have used in
class (see me if you have questions).

\begin{enumerate}
\def\labelenumi{\arabic{enumi}.}
\item
  What data are you planning to use for the final project? Describe the
  data: its contents, its format(s), one or more dataframes, etc. If you
  have a link to the dataset, please include it. If you have a file that
  you can upload, please add it to the assignment submission.
\item
  Does this data need to be cleaned in any way to make it tidy? If so,
  what needs to occur?
\item
  Speaking generally, what tasks will you perform with this dataset? It
  might include tasks that we have not yet covered how to accomplish in
  class, but come up with a general plan that you would like to follow.
\item
  One of the requirements of the final project is to make some type of
  plot using \texttt{ggplot2} with the data. What do you plan to plot
  from your data?
\end{enumerate}

\emph{Answer}:

\#I have a data that was collected on climate change perception among
households of low-income communities in Dar es Salaam, It contains
sociodemographic informations, their perception of how climate change
impact water security adaptive strategies. This data was gotten through
questionnaires and have open ended questions, with binary, and norminal
responses coded.

\#The data needs to be cleaned as we dont have a 100\% response rate. I
could either remove missing values or fill them up.

\#The task maybe just to clean it up , set up a directory for both the
raw,cleans and output file path then do some visualizations tat is
reproducible.

\#I can plot the oercentage of respondents response and perceptions to
some of the questions




\end{document}
